%%%%%%%%%%%%%%%%%%%%%%%%%%%%%%%%%%%%%%%%%%%%%%%%%%%%%%%%%%%%%%%%%%%%%%%
% This file contains every macro I've used in my master's thesis.     %
%                                                                     %
% Macros that are used for specific kinds of logics and/or proofs are %
% organised by topic. More general macros appear at the top, such as  %
% typesetting conventions and higher-order functions. The aim is to   %
% end up with a macro file I can use for research in the future with  %
% a consistent style convention, and also to make it easy to restyle  %
% my thesis should the need arise.                                    %
%                                                                     %
% The convention is that commands that are intended to be written     %
% inline, such as logical connectives, get lowercase names. Commands  %
% that refer to a particular mathematical object, such as a set of    %
% atoms, get uppercase names.                                         %
%%%%%%%%%%%%%%%%%%%%%%%%%%%%%%%%%%%%%%%%%%%%%%%%%%%%%%%%%%%%%%%%%%%%%%%

%%%%%%%%%%%%%%%%%%%%%%%%%%%%%%%%%%%%%%%%%%%%%%%%%%
% Typesetting and/or Stylistic Conventions       %
%%%%%%%%%%%%%%%%%%%%%%%%%%%%%%%%%%%%%%%%%%%%%%%%%%

% Page geometry and styling.
\setlength{\parskip}{0.5em}
\setlength{\parindent}{0.5em}

% Mathematics environments.
\newtheorem{lemma}{Lemma}
\newtheorem{theorem}{Theorem}
\newtheorem{definition}{Definition}
\newtheorem{example}{Example}
\newtheorem{corollary}{Corollary}

% Algorithm environments.
% \SetKwInput{KwData}{Input}
% \SetKwInput{KwResult}{Output}

% Font styles for various things.
\newcommand{\textprop}[1]{\textsc{#1}} % entailment/satisfaction properties
\newcommand{\textlog}[1]{\textsc{#1}} % names of logics
\newcommand{\textpred}[1]{\textsf{\upshape#1}} % predicates in examples
\newcommand{\textop}[1]{\textrm{#1}} % mathematical operators
\newcommand{\textcomp}[1]{\textsc{#1}} % complexity classes
\newcommand{\textalgo}[1]{\textsf{#1}} % algorithm names

% Aliases and shorthands for the most common things.
\newcommand{\tp}{\textpred}

%%%%%%%%%%%%%%%%%%%%%%%%%%%%%%%%%%%%%%%%%%%%%%%%%%
% Global Mathematical Objects                    %
%%%%%%%%%%%%%%%%%%%%%%%%%%%%%%%%%%%%%%%%%%%%%%%%%%

% Abbreviated names for various logics.
\newcommand{\Klm}{\textlog{Klm}} % propositional KLM
\newcommand{\Alc}{\textlog{Alc}} % description logic ALC
\newcommand{\Ptl}{\textlog{Ptl}} % propositional typicality logic
\newcommand{\Sql}{\textlog{Sql}} % SQL database query language

% Complexity classes.
\newcommand{\Exptime}{\textcomp{Exptime}} % EXPTIME
\newcommand{\NpComplete}{\textcomp{NP-Complete}} % EXPTIME

% Languages, signatures and sets of symbols.
\newcommand{\Lang}{\ensuremath{\mathcal{L}}} % general language
\newcommand{\Univ}{\mathcal{U}} % set of possible worlds

% Entailment relations and their global properties.
\newcommand{\entails}{\models} % classical entailment
\newcommand{\nentails}{\not\entails}
\newcommand{\dentails}{\mid\hskip-0.40ex\approx} % defeasible entailment
\newcommand{\ndentails}{\not\mid\hskip-0.40ex\approx}
\newcommand{\genentails}{\entails_?} % placeholder entailment relation
\newcommand{\entincl}{\textprop{(Incl)}}
\newcommand{\entidem}{\textprop{(Idem)}}
\newcommand{\entcumu}{\textprop{(Cumu)}}
\newcommand{\entmono}{\textprop{(Mono)}}
\newcommand{\entsmp}{\textprop{(Smp)}}

% Abbreviations for KLM rationality properties
\newcommand{\klmlle}{\textprop{(Lle)}} % left logical equivalence
\newcommand{\klmrw}{\textprop{(Rw)}} % right weakening
\newcommand{\klmrefl}{\textprop{(Refl)}} % reflexivity
\newcommand{\klmand}{\textprop{(And)}} % conjunction in the conclusion
\newcommand{\klmor}{\textprop{(Or)}} % disjunction in the premise
\newcommand{\klmcut}{\textprop{(Cut)}} % cut rule
\newcommand{\klmcm}{\textprop{(Cm)}} % cautious monotonicity
\newcommand{\klmrm}{\textprop{(Rm)}} % rational monotonicity
\newcommand{\klmcons}{\textprop{(Cons)}} % consistency
\newcommand{\klmsc}{\textprop{(Sc)}} % supraclassicality
\newcommand{\klmmon}{\textprop{(Mon)}} % monotonicity
\newcommand{\klmtrans}{\textprop{(Trans)}} % transitivity

% Standard mathematical objects, like numbers.
\newcommand{\Nat}{\mathbb{N}} % naturals
\newcommand{\NatInf}{\mathbb{N}^\infty} % naturals with top element
\newcommand{\Int}{\mathbb{Z}} % integers

% Satisfaction and satisfaction sets.
\newcommand{\SatS}{\mathcal{S}} % satisfaction set
\newcommand{\sat}{\Vdash} % satisfaction relation
\newcommand{\nsat}{\not\sat}

% Ranking functions.
\newcommand{\rk}{\textop{rk}} % ranking function

% Knowledge bases and various functions of them.
\newcommand{\KB}{\ensuremath{\mathcal{K}}} % knowledge base
\newcommand{\Mod}[1]{\textop{Mod}(#1)} % models of a KB

%%%%%%%%%%%%%%%%%%%%%%%%%%%%%%%%%%%%%%%%%%%%%%%%%%
% First-Order Logic                              %
%%%%%%%%%%%%%%%%%%%%%%%%%%%%%%%%%%%%%%%%%%%%%%%%%%

% Languages, symbols and signatures.
\newcommand{\LangFol}{\Lang} % first-order language (as in set of all wffs)
\newcommand{\FolLang}{\Sigma} % first-order language (as in signature)
\newcommand{\FolLangVar}{\textsc{var}} % set of variable symbols
\newcommand{\FolLangPred}{\textsc{pred}} % set of predicate symbols
\newcommand{\FolLangConst}{\textsc{const}} % set of constant symbols
\newcommand{\FolLangFunc}{\textsc{func}} % set of function symbols
\newcommand{\FolLangTerms}{\mathcal{T}} % set of terms
\newcommand{\folLangArity}[1]{\mathfrak{ar}(#1)} % arity of predicate symbol

% Standard semantics
\newcommand{\FolI}{\mathcal{I}} % first-order structure
\newcommand{\FolISet}{\mathscr{I}} % set of all first-order structures
\newcommand{\FolDom}[1]{\Delta^{#1}} % first-order domain
\newcommand{\FolFun}[2]{{#2}^{#1}} % first-order interpretation function
\newcommand{\FolVal}{\upsilon} % first-order valuation function

% Herbrand semantics.
\newcommand{\HerbU}{\mathbb U} % herbrand universe, i.e. set of terms
\newcommand{\HerbB}{\mathbb B} % herbrand base, i.e. set of ground atoms
\newcommand{\HerbI}{\mathcal{H}} % herbrand interpretation
\newcommand{\HerbISet}{\mathbb{H}} % set of all herbrand interpretations

%%%%%%%%%%%%%%%%%%%%%%%%%%%%%%%%%%%%%%%%%%%%%%%%%%
% Propositional KLM                              %
%%%%%%%%%%%%%%%%%%%%%%%%%%%%%%%%%%%%%%%%%%%%%%%%%%

% Propositional KLM versions of languages, signatures etc.
\newcommand{\LangKLM}{\Lang^{\twiddle}} % defeasible propositional formulas
\newcommand{\Prp}{\ensuremath{\mathcal{P}}} % propositional atoms
\newcommand{\Th}[1]{\mathrm{Th}(#1)} % deductive closure

% Simplified propositional logic connectives.
\newcommand{\limp}{\rightarrow} % logical implication
\newcommand{\liff}{\leftrightarrow} % biconditional
\newcommand{\lexor}{\oplus} % exclusive or

% The twiddle symbol for defeasible consequence relations.
\newcommand{\twiddle}{\mathrel|\joinrel\sim}
\newcommand{\ntwiddle}{\not\twiddle}

% Various forms of defeasible entailment.
\newcommand{\prefentails}{\entails_{\PrefI}} % preferential entailment
\newcommand{\rankentails}{\entails_{\RankI}} % ranked entailment
\newcommand{\rcentails}{\dentails_{RC}} % rational closure entailment

% Preferential and ranked interpretations.
\newcommand{\PrefI}{\ensuremath{\mathcal{P}}} % preferential interpretation
\newcommand{\ModI}{\ensuremath{\mathcal{M}}} % modular interpretation
\newcommand{\RankI}{\ensuremath{\mathcal{R}}} % ranked interpretation
\newcommand{\rankleq}{\leq_G} % typicality ordering on ranked interpretations

% Possible worlds and such for satisfaction.
\newcommand{\Poss}[1]{\widehat{#1}} % set of possible worlds for a formula
\newcommand{\Cons}[1]{\Univ^{#1}} % set of possible worlds for an interpretation

% Materialisation and exceptionality for computing rational closure.
\newcommand{\Mat}[1]{\ensuremath{{#1}^{\limp}}}

% Algorithms for computing rational closure.
\newcommand{\BaseRank}[1]{\textop{br}({#1})} % baserank for formulas
\newcommand{\BaseRankAlg}{\textalgo{BaseRank}} % baserank algorithm
\newcommand{\RationalClosureAlg}{\textalgo{RationalClosure}} % rc algorithm

%%%%%%%%%%%%%%%%%%%%%%%%%%%%%%%%%%%%%%%%%%%%%%%%%%
% Defeasible Description Logics                  %
%%%%%%%%%%%%%%%%%%%%%%%%%%%%%%%%%%%%%%%%%%%%%%%%%%

% Languages, signatures and sets of symbols.
\newcommand{\LangALC}{\Lang^\Alc}
\newcommand{\Concepts}{\ensuremath{\mathcal{C}}} % set of concept names
\newcommand{\Roles}{\ensuremath{\mathcal{R}}} % set of role names
\newcommand{\Individuals}{\ensuremath{\mathcal{I}}} % set of individual names
\newcommand{\ABox}{\ensuremath{\mathcal{A}}} % set of a-box statements
\newcommand{\TBox}{\ensuremath{\mathcal{T}}} % set of t-box statements
\newcommand{\DBox}{\ensuremath{\mathcal{D}}} % set of defeasible t-box statements
\newcommand{\LangALCTBox}{\LangALC_\TBox} % set of t-box statements
\newcommand{\LangALCDBox}{\LangALC_\DBox} % set of defeasible t-box statements
\newcommand{\LangALCABox}{\LangALC_\ABox} % set of t-box statements

% Connectives for description logics. 
\newcommand{\dlAnd}{\sqcap} % concept conjunction
\newcommand{\dlOr}{\sqcup} % concept disjunction
\newcommand{\dlNot}{\lnot} % concept negation
\newcommand{\dlEquiv}{\equiv} % concept conjunction
\newcommand{\dlForall}[2]{\forall {#1} . {#2}} % value restriction
\newcommand{\dlExists}[2]{\exists {#1} . {#2}} % existential restriction
\newcommand{\subs}{\sqsubseteq} % concept subsumption
\newcommand{\nsubs}{\not\subs}
\newcommand{\dsubs}{\:\raisebox{0.45ex}{\ensuremath{\sqsubset}}\hskip-1.7ex\raisebox{-0.6ex}{\scalebox{0.9}{\ensuremath{\sim}}}\:} % defeasible subsumption
\newcommand{\ndsubs}{\not\hspace{-1.1mm}\dsubs}

% Interpretations and such.
\newcommand{\DlI}{\ensuremath{\mathcal{I}}} % dl interpretation
\newcommand{\DlFun}[2]{{#2}^{#1}} % dl interpretation function
\newcommand{\DlDom}[1]{\Delta^{#1}} % dl interpretation domain
\newcommand{\DlClassFun}[1]{\DlFun{\DlI}{#1}} % classical interpretation function
\newcommand{\DlClassDom}{\DlDom{\DlI}} % classical domain
\newcommand{\DlPrefFun}[1]{\DlFun{\PrefI}{#1}} % preferential interpretation function
\newcommand{\DlPrefDom}{\DlDom{\PrefI}} % preferential domain
\newcommand{\DlModFun}[1]{\DlFun{\ModI}{#1}} % modular interpretation function
\newcommand{\DlModDom}{\DlDom{\ModI}} % modular domain
\newcommand{\DlRankFun}[1]{\DlFun{\RankI}{#1}} % ranked interpretation function
\newcommand{\DlRankDom}{\DlDom{\RankI}} % ranked domain

% Big models, compatible models, and rankings of models.
\newcommand{\DlBigModel}[1]{\RankI(#1)} % big ranked model of a knowledge base
\newcommand{\DlCompatibleModels}[3]{\mathbb{R}^{#1,#2,#3}} % set of compatible ranked models
\newcommand{\dlrankleq}[1]{\leq_{#1}} % I-compatible ranking
\newcommand{\dlaboxleq}[1]{\leq^{A}_{#1}} % I-compatible A-Box ranking

%%%%%%%%%%%%%%%%%%%%%%%%%%%%%%%%%%%%%%%%%%%%%%%%%%
% Defeasible Datalog                             %
%%%%%%%%%%%%%%%%%%%%%%%%%%%%%%%%%%%%%%%%%%%%%%%%%%

% Classical datalog formulas.
\newcommand{\LangDlog}{\Lang} % classical datalog formulas
\newcommand{\LangDlogFacts}{\Lang_{f}} % classical datalog facts
\newcommand{\LangDlogRules}{\Lang_{\limp}} % classical datalog rules
\newcommand{\DlogLangPredEDB}{\FolLangPred_{E}} % EDB predicate symbols
\newcommand{\DlogLangPredIDB}{\FolLangPred_{I}} % IDB predicate symbols

% Defeasible datalog formulas.
\newcommand{\LangDDlog}{\Lang} % defeasible datalog formulas
\newcommand{\LangDDlogCompounds}{\Lang_{c}} % defeasible datalog compounds
\newcommand{\LangDDlogFacts}{\Lang_{f}} % classical datalog facts
\newcommand{\LangDDlogRules}{\Lang_{\limp}} % classical datalog rules
\newcommand{\LangDDlogDRules}{\Lang_{\dlimp}} % defeasible datalog rules

% Connectives for defeasible implication.
\newcommand{\dlimp}{\leadsto} % defeasible implication ~>
\newcommand{\ndlimp}{\,{\not \rightsquigarrow}\,}
\newcommand{\ldlimp}{\;\reflectbox{$\dlimp$}\;} % defeasible implication <~
\newcommand{\nldlimp}{\;\reflectbox{$\ndlimp$}\;}
\newcommand{\llimp}{\,:\!\!\!-\;} % classical implication :-

% Programs, databases and theories.
\newcommand{\DlogProg}{\mathcal{P}} % Datalog program
\newcommand{\DlogDat}{\mathcal{D}} % Datalog database
\newcommand{\DlogTheory}{\Gamma} % Datalog theory

% Substitutions.
\newcommand{\HerbSub}{\varphi} % substitution
\newcommand{\HerbSubId}{\varphi_\mathfrak{1}} % identity substitution

% Logical semantics and translations.
\newcommand{\DlogToFol}{\textop{Tr}} % datalog to FOL translation

% Enriched versions of things for typicality constants (TODO).
\newcommand{\herbS}{\mathcal{T}} % set of typical constants
\newcommand{\eherbB}{\herbB_\herbS} % enriched herbrand base
\newcommand{\eherbU}{\herbU_\herbS} % enriched herbrand universe
\newcommand{\eherbAll}{\herbAll_\herbS} % set of enriched herbrand ints.
\newcommand{\eherbI}{\mathcal{E}} % enriched herbrand int.
